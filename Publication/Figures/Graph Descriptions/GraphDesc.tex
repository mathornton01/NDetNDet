%Stuck-at-patterns V.S. Cell-Aware Top 

%non functional requirements are those that can be used to judge the operation of a system 

%One of the first things that the graph above illustrates is that the number of non-functional test patterns decreases with as the 
%number of repetetive detects. This means that although n-detect might be helpful in order to determine the number of functional faults
%that the number of non-functional faults cannot be detected as easily by increasing the numbe of redundant fault checks. Thus, it becomes 
%aparant that the lower the number of redundant checks in the n-detect pattern set this higher the number of patterns that test the non-functional
%faults. So in order to test for non-functional faults the pattern set size, along with the test time, can be decreased by using no redundant detection.
%In other words using n0 can increase the non-functional coverage while also cutting the costs of using more redundancy with a higher n-detect pattern set. 

%%%%% All Cell-aware faults here we mentioned are not detected by stuck-at ATPG
% Page 1:
%  (1) two fault set and two differnet 10k Good-state patterns got similar results:
%  (2) # of  Non-functional patterns are more than # of functional patterns obviously, even compare with # of stuck at ATPG, it also need to add more patterns obviously, 
%  (3) but # of functional patterns V.S. # of stuck at ATPG are not very siginificantly
%  (4) Hence, we drop Non-functional patterns only add Functional patterns, we can save siginificantly costs ( time and money)
%  (5) With Multi-N-detecton functon increase,  # of  Non-functional patterns is decreasing, but not siginificantly, so,  Multi-N-detecton is helpful for saving Cell-aware fault patterns, but not too much


% Page 2:
%  (1) Equation %= # of non_functional Cell_Aware faults patterns/(stuck_at ATPG + # of funtional Cell_aware patterns)
%  (2) two fault set and two differnet 10k Good-state patterns got similar results:
% (3) With Multi-N-detecton functon increase,  % of  Non-functional cell-aware test patterns is decreasing, but when n = 3, it still remains 3%~4%, and also this decreasing is coming from the number of stuck-at ATPG increas, its real number of patterns are not dereased siginificantly (we can see in Page1)

% Page 3:
%  (1) Equation
%       (a)non_functional = # of non_functional Cell_Aware faults /(# of non_functional Cell_Aware faults+ # of functional Cell_Aware faults)
%        (b)functional = # of functional Cell_Aware faults /(# of non_functional Cell_Aware faults+ # of functional Cell_Aware faults)
%  (2) two fault set and two differnet 10k Good-state patterns got similar results:
%  (3) the non-functional Cell-Aware faults accounted for a large proportion of total of  Cell-aware faults which not detected by stuck-at ATPG
%  (4) The worst case on s12307, it accounts 100%~97%
%  (5) The best case on s38584, it still accounts 50%~55%

% From those figures(Page1~Page3), drop non-fuctional Cell-Aware faults can siginificantly enhance faults testing efficiency

%Page4, the table1
% (1) This table shows # of the funtional faults in each cicuit
% (2) The bigger circuit, the more functional faults

% Other figures in page4 and page 5
% (1) those figures are shown the number of detections for each fault mentioned in Table1
% (2) The X axis represents the each functional fault  (The order is AND=>NAND=>NOR=>OR, the input pins 2=>3=>4)
% (3) some faults are detected very lowest(1~2 times), some are largest (8000 times), those larger number of detected faults are most important for testing
% (4) Multi- N- dectection funciton cannot detected many functional faults even n increase, some of those faults are most important for testing





%%%%%%%%%%%%%%%%%%%%%%PAGE 1%%%%%%%%%%%%%%%%%%%%%%%%%%%%%%%%%%%%%%%%%%%%%%

\begin{document}{}

In the above figures we can see that two different generations of 10\,000 good-states used on two different permutations of Cell-aware fault set yielded similar results. Another observation is that in each of the four cases there are more patterns that test for non-functional defects than those that tests for functional defects that are generated by cell aware ATPG, even in comparison with the large number of patterns generated by stuck-at ATPG, the number of non-functional patterns is significant. The number of functional test patterns is negligible compared to the large amount of patterns generated by the stuck-at ATPG. Because adding all of the non-functional tests to the test set adds significant costs, these figures suggest that using n-detect with a higher degree of n can decrease the number of non-functional tests which will decrease the added costs associated with testing all of the non-functional defects. 

These graphs represent the total percentage of patterns that test for non-functional defaults generated by Cell-aware ATPG out of all Stuck-at ATPG plus functional Cell-aware ATPG patterns for each of the five circuits that were tested. Here again we see that two different sets of 10\,000 good-states, as well as two different Cell-aware fault set's achieved similar results. As the n-detect redundancy increases we see a sharp decrease in the percentage of patterns that check for non-functional defects. Although even when we use n3 there remains 3~4\% of patterns testing non-functional defects this is still a dramatic decrease from the normal percentage which is around 17~18\% of all test patterns. This percentage decrease is not due to a decrease in the actual number of non-functional test patterns (as seen in figure 1), but an overall increase in the number of stuck-at ATPG patterns generated in response to the desired redundancy of the n-detect pattern generation. 

The above figures show the distribution of percentages of functional, and non-functional Cell-aware faults not detected by stuck-at ATPG. Once again the results were similar for all of the tested patterns and Cell-aware fault sets. Different circuits obtained different distributions based on circuits’ characters. However we can see that the non-functional Cell-aware faults that were detected accounted for a large percentage of the total Cell-aware faults for those not detected by Stack-at ATPG . In s13207 the non-functional Cell-aware faults accounted for 97-100\% of all non-detected Cell-aware faults by Stuck-at ATPG, even in s38584 non-functional Cell-aware faults accounted for 50-54\% of the total non-detected Cell-aware faults by Stuck-at ATPG. 

This table shows the number of functional Cell-aware faults in each circuit, because Cell-aware faults are inserted in each of the gates we can see that the larger the circuit, the more functional Cell-aware faults are inserted.

Here are charts for each of the five circuits that were tested. They show the number of functional faults that were detected based on Cell-aware fault sets used and the iteration of the test. The bars above each of the n-detect numbers represent the faults detected in each set of gates: AND, NAND, NOR, and OR respectively, with 2,3, and 4 input pins each. We noticed that some faults were detected only 2 or 3 times, whereas others were detected as many as 8\,000 times. The faults that were detected many times are likely the most crucial faults. N-detect does not detect many important functional faults, even as n increases. This suggests that increasing the redundancy of the test set with n-detect is ineffective for examining these crucial functional faults. 


\end{document}



















